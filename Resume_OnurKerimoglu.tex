% FortySecondsCV LaTeX template
% Copyright © 2019-2022 René Wirnata <rene.wirnata@pandascience.net>
% Licensed under the 3-Clause BSD License. See LICENSE file for details.
%
% Please visit https://github.com/PandaScience/FortySecondsCV for the most
% recent version! For bugs or feature requests, please open a new issue on
% github.
%
% Contributors:
% https://github.com/PandaScience/FortySecondsCV/graphs/contributors
%
% Attributions
% ------------
% * fortysecondscv is based on the twentysecondcv class by Carmine Spagnuolo
%   (cspagnuolo@unisa.it), released under the MIT license and available under
%   https://github.com/spagnuolocarmine/TwentySecondsCurriculumVitae-LaTex
% * further attributions are indicated immediately before corresponding code


%-------------------------------------------------------------------------------
%                             ADDITIONAL PACKAGES
%-------------------------------------------------------------------------------
\documentclass[
	a4paper,
	% 9pt,
	% sidesectionsize=Large,
	% showframes,
	% vline=2.2em,
	% maincolor=cvgreen,
	% sidecolor=gray!50,
	% sidetextcolor=green,
	% sectioncolor=red,
	% subsectioncolor=orange,
	% itemtextcolor=black!80,
	% sidebarwidth=0.4\paperwidth,
	% topbottommargin=0.03\paperheight,
	% leftrightmargin=20pt,
	% profilepicsize=4.5cm,
	% profilepicborderwidth=3.5pt,
	% profilepicstyle=profilecircle,
	% profilepiczoom=1.0,
	% profilepicxshift=0mm,
	% profilepicyshift=0mm,
	% profilepicrounding=1.0cm,
	% logowidth=4.5cm,
	% logospace=5mm,
	% logoposition=before,
	% sidebarplacement=right,
	% datecolwidth=0.22\textwidth,
]{fortysecondscv}

% fine tune line spacing
% \usepackage{setspace}
% \setstretch{1.1}

\usepackage{enumitem}

% improve word spacing and hyphenation
\usepackage{microtype}
\usepackage{ragged2e}

% uncomment in case you don't want any hyphenation
% \usepackage[none]{hyphenat}

% take care of proper font encoding
\ifxetexorluatex
	\usepackage{fontspec}
	\defaultfontfeatures{Ligatures=TeX}
	% \newfontfamily\headingfont[Path=fonts/]{segoeuib.ttf} % use local font
\else
	\usepackage[utf8]{inputenc}
	\usepackage[T1]{fontenc}
\fi

% use a sans serif font as default
\usepackage[sfdefault]{ClearSans}
% \usepackage[sfdefault]{noto}

% multi-language CV XeLaTeX and polyglossia (should also work with LuaLaTeX)
% NOTE: breaks \pointskill, \membership and some spacings
% \ifxetexorluatex
% 	\usepackage{polyglossia}
% 	\newfontfamily\arabicfontsf[Script=Arabic,Scale=1.5]{Amiri}
% 	\newfontfamily\englishfontsf{Clear Sans}
% 	\setmainfont{Amiri}
% 	\setdefaultlanguage{arabic}
% 	\setotherlanguage{english}
% \fi

% enable mathematical syntax for some symbols like \varnothing
\usepackage{amssymb}

% bubble diagram configuration
\usepackage{smartdiagram}
\smartdiagramset{
	% default font size is \large, so adjust to harmonize with sidebar layout
	bubble center node font = \footnotesize,
	bubble node font = \footnotesize,
	% default: 4cm/2.5cm; make minimum diameter relative to sidebar size
	bubble center node size = 0.4\sidebartextwidth,
	bubble node size = 0.25\sidebartextwidth,
	distance center/other bubbles = 1.5em,
	% set center bubble color
	bubble center node color = maincolor!70,
	% define the list of colors usable in the diagram
	set color list = {maincolor!10, maincolor!40,
	maincolor!20, maincolor!60, maincolor!35},
	% sets the opacity at which the bubbles are shown
	bubble fill opacity = 0.8,
}

%-------------------------------------------------------------------------------
%                            PERSONAL INFORMATION
%-------------------------------------------------------------------------------
%% mandatory information
% your name
\cvname{Dr. Onur Kerimoglu}
% job title/career
\cvjobtitle{Data Scientist}
%,\\[0.2em] Panda of the Year}

%% optional information
% profile picture
\cvprofilepic{pics/A2_Onur_014-Bearbeitet-3_400x600.jpg}
% logo picture
%\cvlogopic{pics/logo_txt.png}

% NOTE: ordering in sidebar will mimic the following order
% date of birth
%\cvbirthday{Born: 13.02.1982, Elbistan, Turkey}
% any other custom entry
\cvcustomdata{\faFlag}{Citizenship: Germany}
% short address/location, use \newline if more than 1 line is required
\cvaddress{Beltgens Garten 12\newline20537 Hamburg}
% phone number
\cvphone{+49 176 72375614}
% email address
\cvmail{kerimoglu.o@gmail.com}
% personal website
\cvsite{https://sites.google.com/view/onur-kerimoglu}{sites.google.com/view/onur-kerimoglu}
% pgp key
%\cvkey{4096R/FF00FF00}{0xAABBCCDDFF00FF00}

% include gosquare national flags from https://github.com/gosquared/flags;
% naming according to ISO 3166-1 alpha-2 country codes
\graphicspath{{pics/flags/shiny/}}
	
%-------------------------------------------------------------------------------
%                              SIDEBAR 1st PAGE
%-------------------------------------------------------------------------------
% add more profile sections to sidebar on first page
\addtofrontsidebar{
	
	% social network accounts incl. proper hyperlinks
	\sidesection{Social Network}
		\begin{icontable}{2.5em}{1em}
			%\social{\faLeaf}
			%	{https://de.overleaf.com/latex/templates/forty-seconds-cv/pztcktmyngsk}
            %	{Overleaf Template Link}
            \social{\faLinkedin}
				{https://linkedin.com/in/onur-kerimoglu}
            	{linkedin.com/in/onur-kerimoglu}
			\social{\faGithub}
				{https://github.com/OnurKerimoglu}
				{github.com/OnurKerimoglu}
		\end{icontable}
    
    \sidesection{Technical Skills}
        \skill{\faChevronRight}{Maths}
           \skill[2mm]{\faCaretRight}{\footnotesize diff. calculus, linear algebra, statistics}
        \skill{\faChevronRight}{Programming Languages}
           \skill[2mm]{\faCaretRight}{\footnotesize Python, R, SQL, Fortran (2008), MATLAB}
        \skill{\faChevronRight}{Machine Learning} 
          \skill[2mm]{\faCaretRight}{\footnotesize \footnotesize SciPy, Scikit-learn, PyTorch, TensorFlow,  Keras}
        \skill{\faChevronRight}{Development \& MLOps}
           \skill[2mm]{\faCaretRight}{\footnotesize Git, Shell scripting, FastAPI, Docker, AWS}
        \skill{\faChevronRight}{Data Management and Processing}
          \skill[2mm]{\faCaretRight}{\footnotesize \footnotesize MySQL, Pandas, NumPy}
        %\item Computing Environments 
        %  \vspace{-2mm} \begin{itemize}[label={}] \item {\footnotesize Experience with HPC clusters} \end{itemize} 
        %\item Statistics
        %  \vspace{-2mm} \begin{itemize}[label={}] \item {\footnotesize GLMs, EOF approaches} \end{itemize} 
        \skill{\faChevronRight}{Data Visualization and Reporting}
          \skill[2mm]{\faCaretRight}{\footnotesize Matplotlib, Seaborn, Plotly, \LaTeX} %HTML, CSS, D3, Dash
        \skill{\faChevronRight}{The office suit}
          \skill[2mm]{\faCaretRight}{\footnotesize MS Excel, Word, PowerPoint } %HTML, CSS, D3, Dash  
        %\item Project Managament
        %  \vspace{-2mm} \begin{itemize}[label={}] \item {\footnotesize Trello} \end{itemize}  

    \sidesection{Languages}
    \pointskill{\flag{TR.png}}{Turkish}{5}
    \pointskill{\flag{GB.png}}{English}{5}
    \pointskill{\flag{DE.png}}{German}{4}
    \pointskill{\flag{FR.png}}{French}{2}

%     \sidesection{Soft Skills}
% 		\skill{\faChevronRight}{Analytical thinking}
% 		\skill{\faChevronRight}{Problem solving}
% 		\skill{\faChevronRight}{Communication}
% 		\skill{\faChevronRight}{Adaptability}
% 		\skill{\faChevronRight}{Leadership, Supervision}

}


%-------------------------------------------------------------------------------
%                              SIDEBAR 2nd PAGE
%-------------------------------------------------------------------------------
\definecolor{pastelgreen}{HTML}{D7ECD9}
\definecolor{pastelpurple}{HTML}{D5D6EA}
\definecolor{pastelorange}{HTML}{F5D5CB}
\definecolor{pastelyellow}{HTML}{F6F6EB}

\addtobacksidebar{
    
% 	\sidesection{About Me}
% 	\aboutme{
% 		The giant panda is a terrestrial animal and primarily spends its life
% 		roaming and feeding in the bamboo forests of the Qinling Mountains and in
% 		the hilly province of Sichuan.
% 	}

% 	\sidesection{Diagrams}
% 	\begin{sidebarminipage}
% 		\chartlabel[pastelgreen]{Bubble}
% 		\chartlabel[pastelgreen]{Diagrams}
% 		\chartlabel[pastelpurple]{with}
% 		\chartlabel[pastelpurple]{proper}
% 		\chartlabel[pastelorange]{overflow}
% 		\chartlabel[pastelorange]{protection}
% 		\chartlabel[pastelyellow]{for}
% 		\chartlabel[pastelyellow]{labels}
% 	\end{sidebarminipage}
% 
% 	\begin{figure}\centering
% 		\smartdiagram[bubble diagram]{
% 			\textcolor{white}{\textbf{Being a}} \\
% 			\textcolor{white}{\textbf{Panda}}, % center bubble
% 			\textcolor{black!90}{Eating},
% 			\textcolor{black!90}{Sleeping},
% 			\textcolor{black!90}{Rolling},
% 			\textcolor{black!90}{Playing},
% 			\textcolor{black!90}{Chilling}
% 		}
% 	\end{figure}
% 
% 	\chartlabel{Wheel Chart}
% 
% 	\wheelchart{3.7em}{2em}{%
% 	20/3em/maincolor!50/Chill,
% 	15/3em/maincolor!15/Play,
% 	30/4em/maincolor!40/Sleep,
% 	20/3em/maincolor!20/Eat
% 	}
% 
% 	\sidesection{Barskills}
% 	\barskill[1ex]{\faSkyatlas}{Wearing asian rice hats}{60}
% 	\barskill[2ex]{\faImage}{Playing Chess}{30}
% 	\barskill[3ex]{\faMusic}{Playing the bamboo flute}{50}
    
        
% 	\sidesection{Memberships}
% 	\begin{memberships}
% 		\membership[4em]{pics/logo.png}{PandaScience.net}
% 		\membership[4em]{pics/logo.png}{Some longer text spanning over more than
% 			only one line}
% 	\end{memberships}
}


%-------------------------------------------------------------------------------
%                         TABLE ENTRIES RIGHT COLUMN
%-------------------------------------------------------------------------------
\begin{document}

\makefrontsidebar

\cvsection{Profile Summary}
I am a data scientist with expertise in prototyping and productionizing machine learning models.
I have 15+ years experience in programming, building platforms for wrangling, analysis and visualization of data, and developing mathematical and statistical models.
Having published 30+ scientific publications and 90+ presentations, and having led several research projects in academia and industry, I am comfortable in managing projects, supervising team members, dealing with complex problems and communicating them in interdisciplinary environments.
As a continuous learner, I enjoy discovering current technologies and developing principled AI solutions to various technical challenges.

% \cvsection{section}
% \cvsubsection{Subsection}
% \begin{cvtable}
% 	\cvitem{<dates>}{<cv-item title>}{<location>}{<optional: description>}
% \end{cvtable}
 
% \cvsection{Example Data Science Projects}
% \href{https://medium.com/@kerimoglu.o}{\color{pblue}{Medium Profile}}: Blogposts on various data science topics\\
% \vskip -1em
% \href{https://onurkerimoglu.github.io/}{\color{pblue}{GitHub.io Pages}}: Quarto blogs transformed from Jupyter Notebooks\\
% \vskip -1em
% \href{https://huggingface.co/spaces/OnurKerimoglu/facemoods}{\color{pblue}{FaceMoods}}: predicting emotions in human faces detected in an image using an OpenCV face-detection model chained to a pretrained image classification model fine-tuned with Keras \& TensorFlow.\\
% \vskip -1em
% \href{https://onurkerimoglu.pythonanywhere.com}{\color{pblue}{AccIndex}}: predicting accident frequencies on the streets of Berlin, using a Graph NN model built using PyTorch, based features like network structrue, max. speed, population density, area use, etc.

\cvsection{Work Experience}
\begin{cvtable}[1.0]
    
    \cvitem{03.2023 -- present}{Senior Data Scientist}
	{\href{https://datalogue.com}{\color{pblue}{Datalogue GmbH, Germany}}}
	{\textit{Tasks}: 
    \begin{itemize}[topsep=0pt,itemsep=0pt,partopsep=0pt, parsep=0pt, leftmargin=*]
     \setlength\itemsep{1mm}
     \item Overseeing the daily tasks of a team of 7 data scientists, data \& ML engineers
     \item Planning sprints, leading scrum meetings, managing backlog
     \item Aligning with the cross-functional teams for the resolution of bugs and implementation of new features
     \item Monitoring progress and challenges and communicating these with the project managers
     \item Contributing to the product development and prototyping and productionizing ML-based solutions   
    \end{itemize}
    \textit{Tech stack:} Jira, Google Bigquery \& Vertex AI, Python Pandas, faiss, spacy NLP models\\
    }
    
    \cvitem{01.2023 -- 03.2023}{Personal Projects}
	{}
	{\textit{Activities}: 
    \begin{itemize}[topsep=0pt,itemsep=0pt,partopsep=0pt, parsep=0pt, leftmargin=*]
     \setlength\itemsep{1mm}
     \item Prototyping/deploying ML models, writing blogposts
    \end{itemize}
    \textit{Tech stack:} SQL, Pandas, TensorFlow, Keras, Docker\\
    \textit{Open-source software and blogposts}:\\
    \href{https://huggingface.co/spaces/OnurKerimoglu/facemoods}{\color{pblue}{FaceMoods}}: predicting emotions on human faces detected in images using CV models \\ %an object detection model chained to a fine-tuned image classification model \\ %using an OpenCV face-detection model chained to a pretrained image classification model fine-tuned with Keras \& TensorFlow.\\
    \href{https://medium.com/@kerimoglu.o}{\color{pblue}{Medium Profile}}: blogposts on various DS topics\\
    \href{https://onurkerimoglu.github.io/}{\color{pblue}{GitHub.io Pages}}: \href{https://quarto.org/}{\color{pblue}{quarto}} blogs based on Jupyter NB's\\
    }
    
    \cvitem{09.2022 -- 12.2022}{Data Science Bootcamp}
	{\href{https://datascienceretreat.com}{\color{pblue}{DSR, Germany}}}
	{\textit{Tasks}: 
    \begin{itemize}[topsep=0pt,itemsep=0pt,partopsep=0pt, parsep=0pt, leftmargin=*]
     \setlength\itemsep{1mm}
     \item Prototyping/deploying ML models %in solitude and teams
    \end{itemize}
    \textit{Tech stack:} SQL, Pandas, SciPy, Scikit-learn, TensorFlow, Keras, PyTorch, Docker, AWS\\
    \textit{Open-source software produced}:\\
    \href{https://onurkerimoglu.pythonanywhere.com}{\color{pblue}{AccIndex}}: predicting accident frequencies on the streets of Berlin using a Graph NN model % built using PyTorch, based features like network structrue, max. speed, population density, area use, etc.
    }
    
	\cvitem{10.2019 -- 09.2022}{Project Leader}
	{\href{https://uol.de/en/icbm}{\color{pblue}{Uni. Oldenburg, Germany}}}
	{\textit{Tasks}:
	\begin{itemize}[topsep=0pt,itemsep=0pt,partopsep=0pt, parsep=0pt, leftmargin=*]
     \item Leading a project funded by \href{https://www.dfg.de/en}{\color{blue}{DFG}} on development of a bleeding-edge mathematical model of phytoplankton %growth, in collaboration with colleageus in JSPS (Japan), GEOMAR (Germany) and Uni. Liverpool (UK) \item Extension of a model for simulating a lab experiment on microorganismal growth, coagulation/flocculation and nutrient recycling processes  % in collaboration with colleagues in ICBM 
     \item Adaptation of a numerical model for simulating the biogeochemistry of a coastal system.
     \item Supervision of two postdoctoral researchers
    \end{itemize}
      \textit{Tech stack:} Python, R, Fortran, git, Linux, cloud computing, MS Office, \LaTeX\\  
      \textit{Open-source software produced}:\\
      \href{https://github.com/OnurKerimoglu/fabm-nflexpd}{\color{pblue}{FABM-NflexPD}}: A consistent and efficient acclimative model of phytoplankton growth\\
      \href{https://github.com/OnurKerimoglu/R-EcoMol}{\color{pblue}{R-EcoMol}}: A set of R scripts to read, process and visualize lab experiments.\\
      \textit{Papers}:
      \href{https://doi.org/10.1016/j.ecolmodel.2020.109401}{\color{pblue}{1}}, %Krishna et al
      \href{https://doi.org/10.1111/ele.13680}{\color{pblue}{2}},  %Ryabov et al 2021
      \href{https://doi.org/10.5194/gmd-14-6025-2021}{\color{pblue}{3}}, %Kerimoglu et al 2021
      \href{https://doi.org/10.3389/fmars.2021.675428}{\color{pblue}{4}}, %Anugerahanti et al 2021
      \href{https://doi.org/10.1002/lno.12005}{\color{pblue}{5}}, %Hillebrand et al 2022
      \href{https://doi.org/10.3389/fmars.2022.975414}{\color{pblue}{6}}, %Acevedo-Trejos et al 2023
      \href{https://doi.org/10.1016/j.scitotenv.2022.158757}{\color{pblue}{7}}, %Tilstone et al 2023
      \href{https://ospar.org/documents?v=48846}{\color{pblue}{8}}, %OSPAR report 2022
      \href{https://doi.org/10.5194/gmd-16-95-2023}{\color{pblue}{9}}, %Kerimoglu et al 2023
      \href{https://doi.org/10.3389/fmars.2022.963325}{\color{pblue}{10}}, %Singer et al 2023
      \href{https://doi.org/10.1101/2022.05.18.492269}{\color{pblue}{11}} %Kerimoglu et al BioRxiv
    }
    
\end{cvtable}


\newpage
%\makebacksidebar

\newgeometry{
	top=0.04\paperheight,
	bottom=0.04\paperheight,
	right=0.04\paperwidth,
	left=0.04\paperwidth
}

%\setlength{\sidebarwidth}{0.38\paperwidth}
%\setlength{\topbottommargin}{0.02\paperheight}
%\setlength{\leftrightmargin}{0.02\paperwidth}
% \newgeometry{
% 	top=\topbottommargin,
% 	bottom=\topbottommargin,
% 	right=\leftrightmargin,
% 	left=\leftrightmargin
% }

% name and job
%\nameandjob
%\vspace*{-20mm}
%just the name
{\Huge\color{maincolor}\cvname}
\vspace*{-10mm}

\cvsection{}
\cvsubsection{Work Experience (continued)}
\begin{cvtable}[1.0]

\cvitem{07.2018 -- 09.2019}{Project Leader}
	{\href{https://www.hereon.de/}{\color{pblue}{Helmholtz-Zentrum Hereon, Germany}}}
	{\textit{Tasks}:
	\begin{itemize}[topsep=0pt,itemsep=0pt,partopsep=0pt, parsep=0pt, leftmargin=*]
     \item Leading a project funded by \href{https://www.umweltbundesamt.de/en}{\color{blue}{UBA}} on development and application of a 3-D physical-biogeochemical model %, and its application on the North Sea
        %\item Preparation of interim-reports in german
        \item Supervision of a postdoctoral scientist
    \end{itemize}
      \textit{Tech stack:} Python, Fortran, git, Linux, cloud computing, MS Office, \LaTeX\\
      \textit{Open-source software produced}:\\
      \href{https://github.com/OnurKerimoglu/FABM-GPM}{\color{pblue}{FABM-GPM}}: A modular model that allows run-time configuration of food-web sub-models\\
      %\href{git@github.com:OnurKerimoglu/3Dval.git}{\color{pblue}{3Dval}}: A set of Python scripts for the validation and visualization of 3-D biogeochemical models\\
      \textit{Papers}:
      \href{https://doi.org/10.3389/fmars.2019.00370}{\color{pblue}{1}}, %van Beusekom et al 2019
      \href{https://doi.org/10.1029/2019JC015987}{\color{pblue}{2}}, %Chegini et al 2020
      \href{https://doi.org/10.5194/bg-17-5097-2020}{\color{pblue}{3}}, %Kerimoglu et al 2020
      \href{https://doi.org/10.3389/fmars.2021.596126}{\color{pblue}{4}}, %Friedland et al. 2021
      \href{https://doi.org/10.3389/fmars.2021.637483}{\color{pblue}{5}} %Stegert et al. 2021
      }
      
\cvitem{05.2013 -- 06.2018}{Postdoctoral Researcher}
	{\href{https://www.hereon.de/}{\color{pblue}{Helmholtz-Zentrum Hereon, Germany}}}
	{\textit{Tasks}:
	\begin{itemize}[topsep=0pt,itemsep=0pt,partopsep=0pt, parsep=0pt, leftmargin=*]
     \item Development of a novel 3D physical-biogeochemical model abd its evaluation against observation datasets
     %\item Compilation of a multi-compoenent observation dataset consisting of satellite, cruise and station data
     %\item Application of the model to the North Sea and its evaluation against observation datasets
     \item Supervision of three students ranging from MS to PhD level
    \end{itemize}
      \textit{Tech stack:} Python, Fortran, git, Linux, cloud computing, MS Office, \LaTeX \\
      %Open-source software contributed:
      %\href{git@github.com:platipodium/mossco-code.git}{\color{pblue}{MOSSCO}} (Modular System for Shelves and Coasts): a framework for coupling processes or domains that are originally developed in standalone numerical models.\\
      \textit{Papers}:
      \href{https://doi.org/10.3389/fevo.2016.00131}{\color{pblue}{1}}
      \href{https://doi.org/10.1016/j.ecolmodel.2017.07.008}{\color{pblue}{2}},
      \href{https://doi.org/10.5194/bg-14-4499-2017}{\color{pblue}{3}},
      \href{https://doi.org/10.5194/gmd-11-915-2018}{\color{pblue}{4}},
      \href{https://doi.org/10.1016/j.ecss.2017.11.002}{\color{pblue}{5}},
      \href{https://doi.org/10.1007/s10750-018-3653-5}{\color{pblue}{6}}
      }

    \cvitem{05.2012 -- 04.2013}{Postdoctoral Researcher}
	{\href{https://www.inrae.fr/en}{\color{pblue}{INRAe, France}}}
	{\textit{Tasks}:
	\begin{itemize}[topsep=0pt,itemsep=0pt,partopsep=0pt, parsep=0pt, leftmargin=*]
     %\item Setting up a 1-D physical model of Lake Bourget
     \item Development a modular ecosystem model, and its application coupled to a 1-D physical model
     \item Wrangling of raw data to form a consistent observation dataset, evaluating model skill 
    \end{itemize}
      \textit{Tech stack:} R, Fortran, git, MS Office, \LaTeX \\
      \textit{Papers}:
      \href{https://doi.org/10.1111/fwb.12444}{\color{pblue}{1}}
      \href{https://doi.org/10.1007/s10021-014-9837-6}{\color{pblue}{2}}
      \href{https://doi.org/10.1016/j.ecolmodel.2017.06.005}{\color{pblue}{3}}
      \href{https://doi.org/10.1016/j.envsoft.2017.11.016}{\color{pblue}{4}}
      }       

    \cvitem{11.2011 -- 04.2012}{Guest Researcher}
	{\href{https://www.ufz.de/en}{\color{pblue}{UFZ, Germany}}}
	{\textit{Tasks}:
	\begin{itemize}[topsep=0pt,itemsep=0pt,partopsep=0pt, parsep=0pt, leftmargin=*]
     \item Setting up and application of a 1-D hdyrodynamical model to a water reservoir
     \item Performing a scenario analysis for the determination of optimal operation of the reservoir
    \end{itemize}
      \textit{Tech stack:} MATLAB, Windows Batch, MS Office \\
      \textit{Papers}:
      \href{https://doi.org/10.1002/2013WR013520}{\color{pblue}{1}}
      }
      
    \cvitem{05.2008 -- 10.2011}{Project assistant}
	{\href{https://www.uni-konstanz.de/}{\color{pblue}{Uni. Konstanz, Germany}}}
	{\textit{Tasks}:
	\begin{itemize}[topsep=0pt,itemsep=0pt,partopsep=0pt, parsep=0pt, leftmargin=*]
     \item Developing a 1-D coupled physical-biological model of the Lake Constance
     %\item Model-based analysis of phytoplankton growth in the lake
     \item Theoretical studies on competition of phytoplankton for resources 
    \end{itemize}
      \textit{Tech stack:} MATLAB, MS Office \\
      \textit{Papers}:
      \href{https://doi.org/10.1111/j.1365-2486.2009.02158.x}{\color{pblue}{1}},
      \href{https://doi.org/10.1016/j.jtbi.2012.01.044}{\color{pblue}{2}},
      \href{https://doi.org/10.1007/s12080-012-0164-2}{\color{pblue}{3}},
      \href{https://doi.org/10.1111/j.1600-0706.2012.20603.x}{\color{pblue}{4}},
      \href{https://doi.org/10.1007/s10750-013-1551-4}{\color{pblue}{5}},
      \href{https://doi.org/10.1890/14-0839.1}{\color{pblue}{6}}
      }
      
    \cvitem{06.2008 -- 05.2011}{Project Assistant}
	{\href{https://https://www.metu.edu.tr/}{\color{pblue}{METU, Turkey}}}
	{\textit{Tasks}:
	\begin{itemize}[topsep=0pt,itemsep=0pt,partopsep=0pt, parsep=0pt, leftmargin=*]
     \item Compilation of hydro-meteorological data sets for lakes
     \item Development of a multi-variate statistical (EOF) model
     \end{itemize}
     \textit{Tech stack:} MATLAB, MS Office, \LaTeX \\
     }
\end{cvtable}

\cvsection{Education and Training}
\begin{cvtable}[1.0]
	\cvitem{09.2022 -- 12.2022}{Data Science Bootcamp}{\href{https://datascienceretreat.com}{\color{pblue}{DSR, Germany}}}
		{Lectures, hands-on projects on various AI subjects and a 3-week final project on development and deployment of ML models, with an emphasis on DL in the context of NLP, computer vision and graph data.}
	\cvitem{05.2008 -- 10.2011}{PhD, Theoretical Ecology}{\href{https://www.uni-konstanz.de/}{\color{pblue}{Uni. Konstanz, Germany}}}
		{\href{https://kops.uni-konstanz.de/bitstream/handle/123456789/16454/Kerimoglu_2011_PhD_Dissertation.pdf?isAllowed=y&sequence=1}{\color{pblue}{Thesis}} on development of a 1-Dimensional coupled physical-biological model for the simulation of Lake Constance and theoretical analysis of phytoplankton competition in heterogeneous environments.}
	\cvitem{09.2005 -- 02.2008}{MSc, Biology}{\href{https://www.metu.edu.tr}{\color{pblue}{METU, Turkey}}}
		{\href{http://etd.lib.metu.edu.tr/upload/12609272/index.pdf}{\color{pblue}{Thesis}} on MV statistical analysis of the relationships between large-scale atmospheric patterns and lake ecosystems. Grad courses incl. statistical learning, time series analysis, CFD, transport phenomena.}
    \cvitem{09.2000 -- 06.2004}{BSc, Environmental Engineering}{\href{https://www.metu.edu.tr}{\color{pblue}{METU, Turkey}}}
		{Courses taken included differential calculus, numerical methods, engineering statistics, algorithms and data structures, environmental modelling and ground water modelling.}
\end{cvtable}


% \cvsection{section}
% \cvsubsection{Subsection}
% \begin{cvtable}
% 	\cvitem{<dates>}{<cv-item title>}{<location>}{<optional: description>}
% \end{cvtable}
% 
% \cvsection{cvitem}
% \cvsubsection{Multi-line with longer description}
% \begin{cvtable}
% 	\cvitem{date}{Description}{location}{Some longer and more detailed
% 		description, that takes two lines of space instead of only one.}
% 	\cvitem{date}{Description}{location}{Some longer and more detailed
% 		description, that takes two lines of space instead of only one.}
% 	\cvitem{date}{Description}{location}{Some longer and more detailed
% 		description, that takes two lines of space instead of only one.}
% \end{cvtable}
% 
% \cvsubsection{One-line without description}
% \begin{cvtable}
% 	\cvitem{Award}{One-line description}{Sponsor}{}
% 	\cvitem{Award}{One-line description}{Sponsor}{}
% 	\cvitem{Award}{One-line description}{Sponsor}{}
% \end{cvtable}
% 
% \cvsection{cvitemshort}
% \cvsubsection{One-line}
% \begin{cvtable}
% 	\cvitemshort{Key}{Some further description}
% 	\cvitemshort{Key}{Some further description}
% 	\cvitemshort{Key}{Some further description}
% \end{cvtable}
% 
% \cvsubsection{Multi-line with longer description}
% \begin{cvtable}
% 	\cvitemshort{Key}{Some further description. Can fill even more than
% 		only one single line while still keeping the correct indendation level.}
% 	\cvitemshort{Key}{Some further description. Can fill even more than
% 		only one single line while still keeping the correct indendation level.}
% 	\cvitemshort{Key}{Some further description. Can fill even more than
% 		only one single line while still keeping the correct indendation level.}
% \end{cvtable}
% 
% \cvsection{cvpubitem}
% \begin{cvtable}
% 	\cvpubitem{Publication title}{A12.2022uthors}{Journal}{Year}
% 	\cvpubitem{Publication title}{Authors}{Journal}{Year}
% 	\cvpubitem{Publication title that is spanning over multiple lines and still
% 		does not look too bad}{Authors}{Journal}{Year}
% \end{cvtable}

% \cvsection{Publications}
% \begin{cvtable}
% 	\cvpubitem{Cooking: 100 recipes for lazy Pandas}{Me and My Panda Friends}
% 		{Panda's Culinary World}{2010}
% 	\cvpubitem{Pandastasia}{Still Me}{Bamboo Books Assoc.}{2005}
% \end{cvtable}

% \cvsection{Awards}
% \begin{cvtable}
% 	\cvitem{2010 -- now}{Panda of the Year}{Panda World Forum}{}
% 	\cvitem{2005 -- now}{Face of World Wide Fund for Nature}{WWF}{}
% 	\cvitem{2000}{Winner of Bamboo Sprouts Eating Contest}{Bamboo Society}{}
% \end{cvtable}
% 
% \cvsection{Extra-Curricular Activities}
% \begin{cvtable}
% 	\cvitemshort{Relaxing}{Master the fine art of relaxing everywhere}
% 	\cvitemshort{Music}{Playing the bamboo flute in the 1st Panda Orchestra}
% 	\cvitemshort{Education}{Teaching young pandas to be more panda-like}
% \end{cvtable}

\cvsignature

\end{document}
